% !TEX TS-program = pdflatex
% !TEX encoding = UTF-8 Unicode
%%
%% This is file `sample-sigconf.tex',
%% generated with the docstrip utility.
%%
%% The original source files were:
%%
%% samples.dtx  (with options: `sigconf')
%% 
%% IMPORTANT NOTICE:
%% 
%% For the copyright see the source file.
%% 
%% Any modified versions of this file must be renamed
%% with new filenames distinct from sample-sigconf.tex.
%% 
%% For distribution of the original source see the terms
%% for copying and modification in the file samples.dtx.
%% 
%% This generated file may be distributed as long as the
%% original source files, as listed above, are part of the
%% same distribution. (The sources need not necessarily be
%% in the same archive or directory.)
%%
%%
%% Commands for TeXCount
%TC:macro \cite [option:text,text]
%TC:macro \citep [option:text,text]
%TC:macro \citet [option:text,text]
%TC:envir table 0 1
%TC:envir table* 0 1
%TC:envir tabular [ignore] word
%TC:envir displaymath 0 word
%TC:envir math 0 word
%TC:envir comment 0 0
%%
%%
%% The first command in your LaTeX source must be the \documentclass command.
\documentclass[sigconf]{acmart}

\settopmatter{printacmref=false} % Removes citation information below abstract
\renewcommand\footnotetextcopyrightpermission[1]{} % removes footnote with conference information in first column
%%\pagestyle{plain} % removes running headers
\thispagestyle{empty}
%%
%% \BibTeX command to typeset BibTeX logo in the docs
\AtBeginDocument{%
  \providecommand\BibTeX{{%
    \normalfont B\kern-0.5em{\scshape i\kern-0.25em b}\kern-0.8em\TeX}}}

%% Rights management information.  This information is sent to you
%% when you complete the rights form.  These commands have SAMPLE
%% values in them; it is your responsibility as an author to replace
%% the commands and values with those provided to you when you
%% complete the rights form.
\setcopyright{none}
\copyrightyear{}
\acmYear{}
\acmDOI{}

%% These commands are for a PROCEEDINGS abstract or paper.
\acmConference[Computer Science]{}{University of Salerno}{UNISA}
\acmBooktitle{Real time Alexa packets profiling analysis}
\acmPrice{}
\acmISBN{}


%%
%% Submission ID.
%% Use this when submitting an article to a sponsored event. You'll
%% receive a unique submission ID from the organizers
%% of the event, and this ID should be used as the parameter to this command.
%%\acmSubmissionID{123-A56-BU3}

%%
%% The majority of ACM publications use numbered citations and
%% references.  The command \citestyle{authoryear} switches to the
%% "author year" style.
%%
%% If you are preparing content for an event
%% sponsored by ACM SIGGRAPH, you must use the "author year" style of
%% citations and references.
%% Uncommenting
%% the next command will enable that style.
%%\citestyle{acmauthoryear}

%%
%% end of the preamble, start of the body of the document source.
\begin{document}

%%
%% The "title" command has an optional parameter,
%% allowing the author to define a "short title" to be used in page headers.
\title{Real time Alexa packets profiling analysis}


%%
%% The "author" command and its associated commands are used to define
%% the authors and their affiliations.
%% Of note is the shared affiliation of the first two authors, and the
%% "authornote" and "authornotemark" commands
%% used to denote shared contribution to the research.
\author{Boi Biagio}
\email{b.boi@studenti.unisa.it}
\affiliation{%
  \institution{Universit\'a degli studi di Salerno}
  \streetaddress{}
  \city{Salerno}
  \state{}
  \country{Italy}
  \postcode{}
}

%%
%% By default, the full list of authors will be used in the page
%% headers. Often, this list is too long, and will overlap
%% other information printed in the page headers. This command allows
%% the author to define a more concise list
%% of authors' names for this purpose.
\renewcommand{\shortauthors}{}

%%
%% The abstract is a short summary of the work to be presented in the
%% article.
\begin{abstract}
Nowadays, the introduction of home virtual assistents like Alexa Echo or Google Home became a practice, just considering that over 27\% of families owns one.

It's obvious that those devices simplified the life by creating a smart house with few money; but what's the impact these devices have on people privacy?
There are a lot of cases in the United States in which the judge asked to Amazon to provide the recording done by the Echo Dot in order to find helpful
evidences for the case; so, the question is: "It's possible to prevent the sending of sensible informations to the servers when the weak word is not pronunced?"

In this project we will profile each packet exchanged between the Alexa Echo and the Server in order to certificate the existence of data sending also when
not expressly requested from the users.
\end{abstract}

%%
%% Keywords. The author(s) should pick words that accurately describe
%% the work being presented. Separate the keywords with commas.
\keywords{data analytics, alexa, packets profiling}

%% A "teaser" image appears between the author and affiliation
%% information and the body of the document, and typically spans the
%% page.

\begin{teaserfigure}
\rule{\linewidth}{1mm}
%%  \includegraphics[width=\textwidth]{sampleteaser}
%%  \caption{Insert text here}
%%  \Description{insert description here}
%%  \label{fig:teaser}
\end{teaserfigure}

%%
%% This command processes the author and affiliation and title
%% information and builds the first part of the formatted document.
\maketitle

\section{Introduction}
TODO

\end{document}
\endinput
%%
%% End of file `sample-sigconf.tex'.
